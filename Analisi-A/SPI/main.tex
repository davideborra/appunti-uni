\documentclass{article}
\usepackage{lambdatex} %disponibile all'indirizzo http://lambdamath.altervista.it/esercizi/lambdatex.sty
\usepackage{tasks}
\usepackage{exsheets}
\newcommand{\se}{\text{ se }}
\renewcommand{\phi}{\varphi}
\everymath{\displaystyle}
\renewcommand{\binom}[2]{\begin{pmatrix}#1\\#2\end{pmatrix}}

\title{Appunti di  Analisi Matematica A\\{\small Seconda Prova Intermedia}}
\author{Davide Borra}
\date{}
\begin{document}
\maketitle
\lhead{Davide Borra}
\rhead{SPI}
\chead{Appunti di Analisi Matematica A}
%\setlength{\headheight}{30pt}
%%%%%%%%%%%%%%%%%%%%%%%%%%%%%%%%%%%%%%%%%%%%%%%%%%%%%
\section*{Sviluppi di Taylor}
Centrati in $x_0=0$
\begin{itemize}
    \item $e^x=1+x+\frac{x^2}{2!}+\frac{x^3}{3!}+\dots+\frac{x^n}{n!}+o(x^n)$
    \item $\log(1+x)=x-\frac{x^2}{2}+\frac{x^3}{3}-\dots+\frac{(-1)^n}{n}x^n+o(x^n)$
    \item $\sen x = x-\frac{x^3}{3!}+\frac{x^5}{5!}-\dots+\frac{(-1)^n}{(2n+1)!}x^{2n+1}$
    \item $\cos x = 1-\frac{x^2}{2}+\frac{x^4}{4!}-\dots+\frac{(-1)^n}{(2n)!}x^{2n}+o\left(x^{2n+1}\right)$
    \item $\senh x = \frac{e^x-e^{-x}}{2}=\sum_{k=0}^n\frac{x^{2k+1}}{(2k+1)!}+o(x^{2n+2})$
    \item $\cosh x = \frac{e^x+e^{-x}}{2}=\sum_{k=0}^n\frac{x^{2k}}{(2k)!}+o(x^{2n+1})$
    \item $\arctg x = x-\frac{x^3}{3}+\frac{x^5}{5}-\dots+\frac{(-1)^nx^{2n+1}}{2n+1}+o(x^{2n+2})$
    \item $\tg x = x+\frac{x^3}{3}+\frac{2}{15}x^5+o(x^6)$
    \item $(1+x)^\alpha=1+\alpha x+\frac{\alpha(\alpha-1)}{2!}x^2+\dots+\frac{\alpha(\alpha-1)\cdot \ldots\cdot (\alpha-(n-1))}{n!}x^n+o(x^n)=\sum_{k=0}^n
    \binom{\alpha}{k}x^k$
\end{itemize}
\subsubsection*{Casi particolari}
\begin{itemize}
    \item[$\Rightarrow$] $\frac{1}{1+x}=1-x+x^2-x^3+\dots+(-1)^nx^n+o(x^n)$
    \item[$\Rightarrow$] $\frac{1}{1-x}=1+x+x^2+x^3+\dots+x^n+o(x^n)$
    \item[$\Rightarrow$] $\sqrt{1+x}=1+\frac{x}{2}-\frac{x^2}{8}+\frac{x^3}{16}+o(x^3)$
\end{itemize}
\section*{Serie}
\subsection*{Serie notevoli}
\begin{itemize}
    \item \textbf{Serie geometrica:} $\sum_{n=0}^{+\infty}q^n$
    \begin{itemize}
        \item Converge se: $|q|<1$ con somma $\frac{1}{1-q}$
        \item Diverge positivamenete se: $q\geq 1$
        \item Indeterminata se:  $q\leq -1$
    \end{itemize}
    \item \textbf{Serie armonica generalizzata:} $\sum_{n=0}^{+\infty}\frac{1}{n^\alpha(\log n)^\beta}$
    \begin{itemize}
        \item Converge se: $\alpha>1, \forall \beta$ oppure $\alpha =1, \beta >1$
        \item Diverge positivamenete se: $\alpha< 1, \forall \beta$ oppure $\alpha =1, \beta \leq 1$
    \end{itemize}
    \item \textbf{Serie telescopiche:} $\sum_{n=0}^{+\infty}a_n=\sum_{n=0}^{+\infty}(b_n-b_{n+1})$\\
    Risulta quindi che la successione delle somme parziali è $s_n=b_0-b_{n+1}$, per cui il carattere della serie dipende da $\lim_{n\to+\infty}b_{n+1}$
\end{itemize}

\subsection*{Criteri di convergenza}
\begin{shadedTheorem}[Condizione necessaria di convergenza]
        Se $\sum_na_n$ è convergente, allora $\lim_{n\to+\infty}a_n=0$
\end{shadedTheorem}
\subsubsection*{Criteri di convergenza per serie a segno costante}
\begin{shadedTheorem}[Confronto]
    Siano $(a_n)_n$ e $(b_n)_n$ due successioni di numeri reali tali che 
    \[0\leq a_n\leq b_n~~~~~~\forall n\in \N\]
    allora
    \begin{enumerate}[label=\roman*\:\textnormal{)},itemindent=*]
        \item $\sum_{n=0}^{+\infty}b_n<+\infty~~\implies~~\sum_{n=0}^{+\infty}a_n<+\infty$
        \item $\sum_{n=0}^{+\infty}a_n=+\infty~~\implies~~\sum_{n=0}^{+\infty}b_n=+\infty$
    \end{enumerate}
\end{shadedTheorem}

\begin{shadedTheorem}[Confronto asintotico]
    Siano $(a_n)_n$ e $(b_n)_n$ due successioni di numeri reali positivi (almeno da un certo $\bar{n}$ in poi) tali che 
    \[\lim_{n\to +\infty}\frac{a_n}{b_n}=l\in \:]0,+\infty[~~~~\left( a_n\asymp b_n \text{ per } n\to+\infty \right)\]
    Allora $\sum_{n=0}^{+\infty}a_n$ e $\sum_{n=0}^{+\infty}b_n$ hanno lo stesso carattere.
\end{shadedTheorem}

\begin{shadedTheorem}[Criterio della radice $n$-esima]
    Sia $(a_n)_n$ una successione di numeri reali non negativi. Se esiste 
    \[\lim_{n\to +\infty }\sqrt[n]{a_n}=l\in \:]0,+\infty[\]
    e 
    \begin{itemize}
        \item $l<1$, allora $\sum_{n=0}^{+\infty}a_n<+\infty$
        \item $l>1$, allora $\sum_{n=0}^{+\infty}a_n=+\infty$
    \end{itemize}
\end{shadedTheorem}
Se $l=1$ non abbiamo informazioni circa il carattere della serie.
\begin{shadedTheorem}[Criterio del rapporto]
    Sia $(a_n)_n$ una successione di numeri reali positivi. Se esiste 
    \[\lim_{n\to +\infty }\frac{a_{n+1}}{a_n}=l\in \:]0,+\infty[\]
    e 
    \begin{itemize}
        \item $l<1$, allora $\sum_{n=0}^{+\infty}a_n<+\infty$
        \item $l>1$, allora $\sum_{n=0}^{+\infty}a_n=+\infty$
    \end{itemize}
\end{shadedTheorem}
Se $l=1$ non abbiamo informazioni circa il carattere della serie.
\subsubsection*{Serie numeriche a segno qualsiasi}
\begin{shadedTheorem}[Criterio dell'assoluta convergenza]
    Sia $(a_n)_n$ una successione di numeri reali positivi. Se la serie $\sum_{n=0}^{+\infty}a_n$ è assolutamente convergente, allora è (semplicemente) convergente, e si ha 
    \[\left|\sum_{n=0}^{+\infty}a_n\right|\leq \sum_{n=0}^{+\infty}\left|a_n\right|\]
\end{shadedTheorem}
\subsubsection*{Serie numeriche a segni alterni}
\begin{shadedTheorem}[Criterio di Leibnitz]
    Sia $(a_n)_n$ una successione di numeri reali tali che :
    \begin{enumerate}[label=\roman*\:\textnormal{)},itemindent=*]
        \item $a_n\geq 0 ~~~~\forall n\in \N$
        \item $(a_n)_n$ decrescente $\forall n\geq\bar n$
        \item $\lim_{n\to +\infty} a_n=0$
    \end{enumerate}
    Allora la serie $\sum_{n=0}^{+\infty}(-1)^na_n$ è convergente.
\end{shadedTheorem}
\subsection*{Teorema di Riemann-Dini}
\begin{shadedTheorem}[Riemann-Dini]
    Se una serie è convergente, ma non assolutamente, allora scelto un qualsiasi $S\in \R$ esiste un riordinamento della serie data con somma $S$. Esistono anche riordinamenti della serie che sono divergenti e altri che sono indeterminati.
\end{shadedTheorem}
\subsubsection*{Serie di potenze}
\begin{shadedTheorem}[determinazione del raggio di convergenza]
    Sia $\sum_{n=0}^{+\infty}a_n(x-x_0)^n$ la serie di potenze data. Se esiste 
    \begin{enumerate}[label=\roman*\:\textnormal{)},itemindent=*]
        \item $\lim_{n\to+\infty}\sqrt[n]{|a_n|}=l\in [0;+\infty]$
    \end{enumerate}
    oppure
    \begin{enumerate}[label=\roman*\:\textnormal{)},itemindent=*]
        \setcounter{enumi}{1}
        \item $\lim_{n\to+\infty}\left|\frac{a_{n+1}}{a_n}\right|=l\in [0;+\infty]$
    \end{enumerate}
    Allora la serie ha raggio di convergenza
    \[r=\begin{cases}
        +\infty &\se l=0\\
        0&\se l=+\infty\\
        \frac{1}{l}&\se 0<l<+\infty
    \end{cases}\]
    Di conseguenza la serie converge in 
    \[E=\begin{cases}
        \R &\se l=0\\
        \{x_0\}&\se l=+\infty\\
        [x_0-l;x_0+l]&\se 0<l<+\infty
    \end{cases}\]
\end{shadedTheorem}
\section*{Integrali generalizzati}
\subsection*{Funzioni illimitate}
\begin{oss}
    Nel caso in cui l'integrale da calcolare sia improprio in entrambi gli estremi bisogna spezzare:
    \[\int_a^bf(x)\d x=\int_a^cf(x)\d x+\int_c^bf(x)\d x=\lim_{\delta_1\to 0}\int_{a+\delta_1}^cf(x)\d x+\lim_{\delta_2\to 0}\int_c^{b+\delta_2}f(x)\d x\]
    Analogamente nel caso in cui si presenta un punto di discontinuità all'interno dell'intervallo di integrazione.
\end{oss}
\paragraph*{Per confrontare:}
\[f(x)\sim \frac{1}{x^\alpha}~~\text{converge se } \alpha <1\]
\subsection*{Intervalli illimitati}
\begin{oss}
    Nel caso in cui l'integrale da calcolare sia improprio in entrambi gli estremi bisogna spezzare:
    \[\int_{-\infty}^{+\infty}f(x)\d x=\int_{-\infty}^af(x)\d x+\int_a^{+\infty}f(x)\d x=\lim_{\delta_1\to -\infty}\int_{\delta_1}^af(x)\d x+\lim_{\delta_2\to -\infty}\int_a^{\delta_2}f(x)\d x\]
\end{oss}
\paragraph*{Per confrontare:}
\[f(x)\sim \frac{1}{x^\alpha}~~\text{converge se } \alpha >1\]

\subsection*{Criteri di convergenza}
Confronto, confronto asintotico (attenzione al segno costante!) e assoluta convergenza funzionano esattamente come con le serie. 
\subsection*{Integrali e serie}
\begin{shadedTheorem}[Criterio integrale per le serie a termini positivi]
    Sia $f:[0;+\infty[\to [0;+\infty[$ decrescente. Poniamo $a_n=f(n)~~\forall n \in \N$
    Allora
    \[\sum_{n=0}^{+\infty}<+\infty~~\Harr~~\int_0^{+\infty}f(x)\d x<+\infty\]
    Inoltre
    \[\sum_{n=1}^{+\infty}a_n\leq \int_0^{+\infty}f(x)\d x\leq \sum_{n=0}^{+\infty}a_n\]
    (vale anche da un certo $\bar n$ in poi)
\end{shadedTheorem}

\section*{Equazioni differenziali}
Notazione: $A(x)=\int a(x)\d x$
\subsection*{Equazioni differenziali del primo ordine a variabili separabili}
\[y'=h(x)g(y)\tag{VS}\]
Troviamo prima di tutto una soluzione particolare costante: poniamo $y'=0$ e ricaviamo $y: g(y)=0$. A questo punto siamo autorizati a dividere per $g(y)$ per unicità della soluzione del problema di Cauchy.ù
\[\frac{y'}{g(y)}=h(x)~~\Harr~~\int\frac{y'}{g(y)}\d x=\int h(x)\d x\]
Sostituimo nel primo integrale $y:=y(x)$, allora $\d y = y'\d x$
\[\Harr ~~ \int \frac{1}{g(y)}\d y\Big|_{y=y(x)}=\int g(x)\d x\]
E determiniamo la primitiva. Invertendo si trova l'integrale generale.
\paragraph*{Caso particolare}\(y'=a(x)y~~~\Harr~~~y=ke^{A(x)}, k\in \R\)
\subsection*{Equazioni differenziali lineari del primo ordine a coefficienti funzioni continue}
\[y=a(x)y+b(x)\tag{L1}\]
Se $b(x)\equiv 0$ l'equazione è a variabili separabili e si dice omogenea. Se $b(x)\neq 0$ l'equazione si dice completa.
Per risolvere:
\begin{enumerate}
    \item Determinare l'integrale generale di $y'=a(x)y$, ovvero $y=ce^{A(x)}$
    \item Cerchiamo una soluzione particolare della completa nella forma $y=c(x)e^{A(x)}$ con $c(x)$ incognito. Troviamo
        \[c(x)=\int b(x)e^{-A(x)}\d x\]
\end{enumerate}
Quindi l'integrale generale è
\[y(x)=ce^{A(x)}+c(x)e^{A(x)}=\left( c+c(x) \right)e^{A(x)}=\boxed{e^{A(x)}\left( c+\int b(x)e^{-A(x)}\d x \right)}\]
\subsection*{Equazioni differenziali del secondo ordine a coefficienti costanti}
\[y''+ay'+by=f(x)\tag{L2}\label{eq:L2}\]
Per risolvere
\begin{enumerate}
    \item Troviamo una soluzione dell'equazione omogenea associata \(y''+ay'+by=0\)
    \begin{enumerate}
        \item Risolviamo in $\C$ l'equazione caratteristica \(z^2+az+b=0\)
        \item Determiniamo le soluzioni fondamentali:
        \begin{center}
            \begin{tabular}[h]{|cc|cc|}
                \hline
                $z_1,z_2\in \R$& distinte &  $y_1(x)=e^{z_1x}$ & $y_2(x)=e^{z_2x}$\\ \hline
                $z\in \R$ &con molteplicità 2 &  $y_1(x)=e^{zx}$ & $y_2(x)=xe^{zx}$\\ \hline
                $z_1,z_2\in \C$ &tali che $z_{1,2}=\alpha \pm i\beta$ &  $~~y_1(x)=e^{\alpha x}\sin \beta x~~$ & $~~y_2(x)=e^{\alpha x}\cos \beta x~~$\\ \hline
            \end{tabular}
        \end{center}
        \item L'integrale generale dell'omogenea è una combinazione lineare delle due
        \[y_{om}(x)=c_1y_1(x)+c_2y_2(x)\]
    \end{enumerate}
    \item[2a.] Cerchiamo una soluzione particolare di \eqref{eq:L2} con il metodo di variazione delle costanti del tipo
    \[\bar{y}(x)=c_1(x)y_1(x)+c_2(x)y_2(x)\]
    omettendo la dimostrazione si ricava 
    \[\begin{cases}
        c_1'(x)y_1(x)+c_2'(x)y_2(x)=0\\
        c_1'(x)y_1'(x)+c_2'(x)y_2'(x)=f(x)\\
    \end{cases}\]
    \item [2b.] Se $f(x)=P(x)e^{\gamma x}\cos(\delta x)$ o $f(x)=P(x)e^{\gamma x}\sin(\delta x)$ dove $P(x)$ è un polinomio di grado $n$. Poniamo $\xi=\gamma+i\delta$.
    \begin{itemize}
        \item Se $\xi$ non è soluzione dell'equazione caratteristica 
        \[\bar y(x)=e^{\gamma x}\left( Q_1(x)\cos \delta x +Q_2(x)\sin \delta x\right)\]
        con $Q_1, Q_2$ polinomi di grado $n$ da determinare imponendo che $\bar y$ soddisfi \eqref{eq:L2}.

        \item Se $\xi$ è soluzione dell'equazione caratteristica con molteplicità 1
        \[\bar y(x)=xe^{\gamma x}\left( Q_1(x)\cos \delta x +Q_2(x)\sin \delta x\right)\]
        con $Q_1, Q_2$ polinomi di grado $n$ da determinare imponendo che $\bar y$ soddisfi \eqref{eq:L2}.
        \item Se $\xi$ è soluzione dell'equazione caratteristica con molteplicità 2
        \[\bar y(x)=x^2e^{\gamma x} Q(x)\]
        con $Q$ polinomio di grado $n$ da determinare imponendo che $\bar y$ soddisfi \eqref{eq:L2}.
    \end{itemize}
\end{enumerate}
In conclusione
\[y(x)=c_1y_1(x)+c_2y_2(x)+\bar y(x)\]
%%%%%%%%%%%%%%%%%%%%%%%%%%%%%%%%%%%%%%%%%%%%%%%%%%%%%
\end{document}
