%Update 27/02/2023
% Changelog
%%%%%%%%%%%%%%%%%%%%%%%%%%%%%%%%%%%%%%%%%%%%%%%%%%%%%%%%%%%%%%%%%%%%%%%%%%%%%%%%%%%%%%%%
\documentclass{article}     %type of document
\usepackage{lambdatex} 

\usepackage{tasks}
\usepackage{exsheets}

\makeatletter
\renewcommand*\env@cases[1][1.2]{%
  \let\@ifnextchar\new@ifnextchar
  \left\lbrace
  \def\arraystretch{#1}%
  \array{@{}l@{\quad}l@{}}%
}
\makeatother

\title{Strumenti informatici per la matematica}
\author{Davide Borra}
\date{}
\makeatletter
\let\runauthor\@author
\let\runtitle\@title
\renewcommand{\sectionmark}[1]{\markright{#1}}
\renewcommand{\footrulewidth}{0.4pt}

\everymath{\displaystyle}
\definecolor{zun}{HTML}{C8D6FD}

\begin{document}

\lhead{}
\chead{}
\rhead{Indice}
\rfoot{\runauthor}

\begin{titlepage}
    \newgeometry{left=3cm, right=3cm, bottom=2cm, top =3cm} 
    \pagestyle{empty}
    \begin{center}
        \vspace*{\fill}
        \vspace{0.5cm}
        \textbf{\Huge \runtitle}\\\vspace{5mm}
        \textsc{\Large \runauthor}
        \vspace{5cm}
    \end{center}
    \vspace*{\fill}
    v. 1.0\\
    \rule{0.8\linewidth}{0.5mm}\\
    {\footnotesize\href{mailto:davide.borra@studenti.unitn.it}{davide.borra@studenti.unitn.it} - \href{http://davideborra.github.io}{davideborra.github.io}}
    \restoregeometry
\end{titlepage}
\thispagestyle{empty}

    \tableofcontents
    \creativecommons

\lhead{\runtitle}
\chead{}
\rhead{\rightmark}
\rfoot{\runauthor}
\newpage
\begin{abstract}
    \LaTeX \ non è un word processor, è un formattatore di testi. Un documento \texttt{.tex} è un file di testo che contiene testo e comandi. Per visualizzare il documento bisogna compilarlo. Esso è lo strumento migliore per la produzione di testi matematici.
\end{abstract}
\section{Introduzione a \LaTeX}
\subsection{Il preambolo}
\paragraph{\textbackslash\texttt{documentclass}} Descrive le regole generali per produrre il documento:
\begin{itemize}
    \item \texttt{article} per articoli scientifici fino a 20 pagine
    \item \texttt{book} per libri più ampi
    \item \texttt{letter} per lettere
    \item \texttt{beamer} per slides
\end{itemize}
Successivamente si ha una parte di documento detta preambolo, dove vengono caricati pacchetti aggiuntivi, specificate alcune impostazioni generali del documento e definiti comandi custom.

Fondamentali nel preambolo sono:
\\ \texttt{ \textbackslash usepackage[utf8]\{inputenc\}\\
\textbackslash usepackage[T1]\{fontenc\}\\
\textbackslash usepackage[italian]\{babel\}\\
\textbackslash usepackage[a4paper, portrait, margin=2cm]\{geometry\}}\\
che nell'ordine definiscono l'encoding del file (in modo che supporti, ad esempio, le lettere accentate), il font in modo da avere font completi, la sillabazione italiana e le parole italiane nei comandi e l'ultimo che definisce la dimensione del foglio.
\subsection{Il documento}
Il corpo del documento va racchiuso tra \texttt{\textbackslash begin\{document\}} e \texttt{\textbackslash end\{document\}}. 

Una cosa interessante da sapere è che \LaTeX\ interpreta il punto come fine della frase, quindi se deve andare a capo preferisce farlo lì. Se vogliamo evitare che questo succeda dobbiamo usare \textasciitilde \ al posto dello spazio.
\subsection{La classe \texttt{letter}}
La classe letter premette di specificare indirizzo, (\texttt{\textbackslash address\{\dots\}}), la firma (\texttt{\textbackslash signature\{\dots\}}) e la data (\texttt{\textbackslash date\{\dots\}}). La lettera deve essere racchiusa tra \texttt{\textbackslash begin\{letter\}\{\textit{\textless destinatario\textgreater}\}} e \texttt{\textbackslash end\{letter\}}.

\subsection{La classe \texttt{article}}
La classe article ha dei comandi per specificare titolo, autore e data. Per creare il titolo con quei parametri si mette il comando \textbackslash\texttt{maketitle}. Inoltre esiste l'ambiente \texttt{abstract} dove va collocato il sommario dell'articolo. L'opzione \texttt{draft} della classe article non carica le immagini e mette in evidenza le righe che escono dai margini. Article ha anche il formato \texttt{twocolumn}

\subsection{Formattazione}
Per formattare il testo si usa \textbackslash\texttt{textbf\{\dots\}} per il grassetto, \textbackslash\texttt{textit\{\dots\}} per il corsivo e \textbackslash\texttt{underline\{\dots\}} per il sottolineato. Inoltre esistono \textbackslash\texttt{textsc\{\dots\}} per il maiuscoletto e \textbackslash\texttt{texttt\{\dots\}} per il monospaziato (typewriter).Un'alternativa è specificare (ad esempio)  all'interno del blocco (racchiuso tra graffe) da mettere in grassetto \{\textbackslash\texttt{bf~\dots}\}. Esiste anche il comando \textbackslash\texttt{em} che mette corsivo o plain in base al contesto. 
\subsubsection{{\color{yellow} I} {\color{orange} c}{\color{red} o}{\color{magenta} l}{\color{purple} o}{\color{blue} r}{\color{teal} i}}
Per poter usare i colori serve il pacchetto \textbackslash\texttt{usepackage[dvipsnames]\{xcolor\}}. Per cambiare il colore si usa \textbackslash\texttt{color\{\textit{\textless nomecolore\textgreater}\}}. Per evidenziare si usa invece \textbackslash\texttt{color\{\textit{\textless nomecolore\textgreater}\}}. Per definire un colore si usa invece \textbackslash\texttt{definecolor\{HTML\}\{{\color{zun}C8D6FD}\}} da mettere nel preambolo.
\subsubsection{Liste}
In \LaTeX\ esistono due tipi di liste:
\begin{itemize}[leftmargin=*]
    \item \texttt{enumerate} per enenchi numerati;
    \item \texttt{itemize} per elenchi puntati.
\end{itemize}
È possibile annidare liste fino a 4 volte anche anternando \texttt{enumerate} e \texttt{itemize}. Ogni riga deve essere preceduta da \textbackslash\texttt{item}.

\[\begin{aligned}
    y'= \lim_{h\to 0} \frac{f(x+h)-f(x_0)}{h}= \lim_{h\to 0}\frac{-(x+h)^2+4(x+h)+x^2-4x}{h}=\\=\frac{\cancel{-x^2}-h^2-2hx+\cancel{4x}+h+\cancel{x^2}-\cancel{4x}}{h}= \lim_{h\to 0}\frac{\cancel{h}(\cancel{-h}-2x+4)}{\cancel{h}}=-2x+4
\end{aligned}\]

\end{document}